
\documentclass[12pt]{article}
\usepackage[utf8]{inputenc}
\usepackage[russian]{babel}
\usepackage{graphicx}
\usepackage{amsmath}
\usepackage{listings}  
\usepackage[a4paper, total={6in, 8in}]{geometry}

\linespread{1.3} % полуторный интервал

\begin{document}
\sloppy

	\begin{titlepage}
		\newpage
		
		\begin{figure}[t]
			\centering
			\includegraphics[width=0.4\textwidth]{mgu}
		\end{figure}
		
		\begin{center}
			Московский государственный университет имени М.В. Ломоносова \\
			Факультет вычислительной математики и кибернетики \\
			Кафедра автоматизации систем вычислительных комплексов \\
		\end{center}
		
		\vspace{6em}
		
		\begin{center}
			\large
			Ермакова Татьяна Ивановна \\
			621
		\end{center}
		
		\begin{center}
			\Large
			\bfseries
			Исследование масштабируемости различных реализаций алгоритмов линейной алгебры
		\end{center}
		
		\vspace{0.3em}
		
		\begin{center}
			\large
			\textsc{
				отчет
			}
		\end{center}
		
		\vspace{\fill}
		
		\begin{center}
			Москва 2018
		\end{center}
		
	\end{titlepage}


	\section{Постановка задачи}
	\textbf{Вариант задания:} 18 \\
	\textbf{Платформа:} Polus, GPU (Magma) \\
	\textbf{Функция:} magma\_dgesv\_gpu \\
	\textbf{Тип данных:} double real precision \\
	\textbf{Задача:} Решение СЛАУ \\
	\begin{equation*}
	Ax=b
	\end{equation*}
	\textbf{Описание реализации:} \\
	Согласно документации MAGMA при решении СЛАУ используется LU-факторизация матрицы A. Таким образом в составе данной функции выполняется две: dgetrf, производящая факторизацию, и dgetrs, решающая уравнение для факторизированной матрицы.
	
	\section{Исследование производительности}
	
	Производительность принято измерять во флопсах (flops) --- количестве операций с плавающей точкой в секунду. Будем использовать гигафлопсы (gflops). \\
	Из документации возьмем оценку для зависимости количества операций от n, где n размер матрицы A: \\ 
	\begin{align*}
	gflop(DGESV) &= \frac{flop(DGETRF) + flop(DGETRS)}{10^{9}} \\
	flop(DGETRF) &= \frac{2}{3}n^{3} - \frac{n^{2}}{2} + \frac{5}{6}n \\
	flop(DGETRS) &= 2n^{2} - n
	\end{align*}
	Итого получим:\\
	\begin{equation*}
	gflops(n) = \frac{\frac{2}{3}n^{3} + \frac{3}{2}n^{2} - \frac{n}{6}}{10^{9}t} \\
	\end{equation*}
	Анализировались матрицы размеров NxN, где $N \in [2000, 10000]$. Для каждого размера производилось 10 запусков, результаты усреднялись. На рисунках ~\eqref{fig100} и ~\eqref{fig200} показаны результаты экспериментов с шагами 100 и 200 соответственно. \\
	
	\begin{figure}[ht]
		\centering
		\includegraphics[width=0.55\textwidth]{hundr}
		\caption{Результаты с шагом 100}
		\label{fig100}
	\end{figure}

	\begin{figure}[ht]
		\centering
		\includegraphics[width=0.55\textwidth]{twohundr}
		\caption{Результаты с шагом 200}
		\label{fig200}
	\end{figure}


	\noindent
	Максимальное значение производительности достигается, как правило, при n = 9900. \\
	
	\noindent
	\large{\textbf{Выводы:}} \\
	По результатам экспериментов можно наблюдать рост производительности. Это достигается засчёт параллельности. За почти то же самое время обрабатываются матрицы больших размеров. С возрастанием размеров возрастает число операций. В документации MAGMA также обращают внимане на то, что оптимальные размеры должны быть кратны 32. Это связано с сутью параллельных процессов на GPU. Поэтому периодически наблюдаются небольшие спады производительности. n = 9900 действительно оказывается максимальным значением для данного диапазона значений с данным шагом. Однако это не значит, что дальше (за отметкой в 10к) производительность сразу начнет падать. (Это было также подтвеждено незначительным числом экспериментов на размерах матриц от 10к до 20к) 
	
	\section{Текст программы}
	\lstset{ language=c,
		%morekeywords={yield,var,get,set,from,select,partial},
		breaklines=true,
		basicstyle=\footnotesize\ttfamily}
	\linespread{1}
	\lstinputlisting{slau.cpp}

	\section{Обоснование корректности результата работы программы:}
	Проверка результата производилась путем перемножения матриц A и X. Результат сравнивался с матрицей B. Для перемножения использовалась библиотека АЛГЛИБ. При расхождении результатов программа логгировала бы ошибку вместо результата работы. Ни в одном из запусков не было обнаружено сообщений об ошибке.
	
	\section{Сведения о программно-аппаратной среде}
	
	\subsection{Компьютер}
	Polus --- параллельная вычислительная система, состоящая из 5 вычислительных узлов. \\

	\noindent
	Основные характеристики каждого узла:
	\begin{itemize}
		\item 2 десятиядерных процессора IBM POWER8 (каждое ядро имеет 8 потоков) всего 160 потоков.
		\item Общая оперативная память 256 Гбайт (в узле 5 оперативная память 1024 Гбайт) с ЕСС контролем.
		\item 2 х 1 ТБ 2.5” 7K RPM SATA HDD.
		\item 2 x NVIDIA Tesla P100 GPU, NVLink.
		\item 1 порт 100 ГБ/сек.
	\end{itemize}
	
	\noindent
	Производительность кластера (Tflop/s): 55,84 (пиковая), 40,39 (Linpack)
	
	\subsection{Компилятор}
	Основной вид комманды: \\
	\centerline{g++ --O3 --DADD\_}
	
	
	\subsection{Библиотеки}
	\begin{itemize}
		\item MAGMA --- для исследуемой функции (+ magma\_lapack генерация).
		\item АЛГЛИБ --- проверка корректности.
	\end{itemize}
	
	\subsection{Прочие сведения}
	\begin{itemize}
		\item Запуск экспериментов через планировщик IBM Spectrum LSF.
		\item python3 и библиотека matplotlib для анализа результатов.
	\end{itemize}
	
\end{document}
