
\documentclass[12pt]{article}
\usepackage[utf8]{inputenc}
\usepackage[russian]{babel}
\usepackage{graphicx}
\usepackage{amsmath}
\usepackage{listings}  
\usepackage[a4paper, total={6in, 8in}]{geometry}

\linespread{1.3} % полуторный интервал

\begin{document}
\sloppy

	\begin{titlepage}
		\newpage
		
		\begin{figure}[t]
			\centering
			\includegraphics[width=0.4\textwidth]{mgu}
		\end{figure}
		
		\begin{center}
			Московский государственный университет имени М.В. Ломоносова \\
			Факультет вычислительной математики и кибернетики \\
			Кафедра автоматизации систем вычислительных комплексов \\
		\end{center}
		
		\vspace{6em}
		
		\begin{center}
			\large
			Ермакова Татьяна Ивановна \\
			621
		\end{center}
		
		\begin{center}
			\Large
			\bfseries
			Исследование масштабируемости решения дифференциального уравнения в трехмерной области
		\end{center}
		
		\vspace{0.3em}
		
		\begin{center}
			\large
			\textsc{
				отчет версия 2
			}
		\end{center}
		
		\vspace{\fill}
		
		\begin{center}
			Москва 2018
		\end{center}
		
	\end{titlepage}


	\section{Математическая постановка задачи}
	\begin{figure}[htbp]
		\centering
		\includegraphics[clip, width=\textwidth, page=1, trim=0cm 3cm 0cm 25cm]{Task2_2.pdf}
		\includegraphics[clip, width=\textwidth, page=2, trim=0cm 16cm 0cm 1cm]{Task2_2.pdf}
	\end{figure}

	\noindent
	\textbf{Вариант задания:} 8 \\
	\textbf{Граничные условия:} Периодические по всем осям \\
	
	\newpage
	\section{Численный метод решения задачи}
	
	\begin{figure}[htbp]
		\centering
		\includegraphics[clip, width=0.85\textwidth, page=2, trim=0cm 32mm 0cm 171mm]{Task2_2.pdf}
	\end{figure}
	\begin{figure}[htbp]
		\centering
		\includegraphics[clip, width=0.85\textwidth, page=2, trim=0cm 2cm 0cm 268mm]{Task2_2.pdf}
	\end{figure}
	\begin{figure}[htbp]
		\centering
		\includegraphics[clip, width=0.85\textwidth, page=3, trim=0cm 17.5cm 0cm 2.1cm]{Task2_2.pdf}
	\end{figure}
	
	\section{Описание программной реализации}
	В качестве аргумента командной строки программе подаётся размер сетки N. \\
	
	\newpage
	\textbf{Основные этапы:}
	\begin{enumerate}
		\item Производится вычисление количества разбиений области по осям.
		\item Процесс вычисляет какая область предназначена ему.
		\item Производится вычисление слоя 0 и обмен границами.
		\item Производится вычисление слоя 1 и обмен границами.
		\item Когда получены два первых слоя начинается последовательное вычисление всех остальных. После вычисления процессами очередного слоя происходит обмен границами. Для обменов используется ассинхронная передача. 
	\end{enumerate}

	Для проверки корректности вычислений ищется наибольшая разница между численным и реальным значением функции решения на последнем слое. Для этого вместо изначальной задачи решается:
	$$\frac{\mathrm d^2 u}{\mathrm d^2 t} = \Delta u + f,$$
	
	В качестве решения была выбрана:
	$$u(t, x, y, z) = (t^2 + 1)\sin(x)\sin(y)\sin(z)$$
	
	При такой функции решения f принимает вид:
	$$f = (3t^2 + 5)\sin(x)\sin(y)\sin(z)$$
	
	В качестве границ области выбрано $2\pi$. Это необходимо для выполнения периодических граничных условий.
	Максимальную разницу между точным и численным решениями каждый процесс считает самостоятельно. После эти разницы собираются в процессе 0 и среди них так же ищется максимум.
	
	\newpage
	\section{Результаты рассчетов}
	
	\textbf{Порядок заначений функции решения в точках, в которых достигается максимальное значение невязки : $10^{-3}$}.\\
	\noindent
	На обоих машинах было выполнено по 20 шагов по временной сетке.
	
	
	\subsection{IBM Polus}
	
	\begin{table}[!h]
		\begin{minipage}{1\linewidth}
			\centering
			\scalebox{0.65}{
				\begin{tabular}{|p{2.1cm}|p{2.3cm}|p{2cm}|p{2.3cm}|p{2.9cm}|}
					\hline
					Число процессоров & Число точек сетки & Время решения &  Ускорение & Невязка \\
					\hline
					1 & $128^3$ &  16.0323 & --- & 0.00009194\\
					2 & $128^3$ &  7.79245 & 2.0 & 0.00009194\\
					4 & $128^3$ &  3.80506 & 2.0 & 0.00009194\\
					8 & $128^3$ &  1.79448 & 2.1 & 0.00009194\\
					\hline
					1 & $256^3$ &  126.139 & --- & 0.00005291\\
					2 & $256^3$ &  62.1457 & 2.0 & 0.0005291\\
					4 & $256^3$ &  31.212 &  2.0 & 0.00005291\\
					8 & $256^3$ &  14.8932 & 2.1 & 0.00005291\\
					\hline
					1 & $512^3$ &  1050.98 & --- & 0.00002807\\
					2 & $512^3$ &  520.7 & 2.0 & 0.00002807\\
					4 & $512^3$ &  246.086 & 2.1 & 0.00002807\\
					8 & $512^3$ &  118.789 & 2.1 & 0.00002807\\
					\hline
				\end{tabular}
			}
		\end{minipage}%
		\caption{Результаты экспериментов IBM Polus}
		\label{table:polus}
	\end{table} 


	\subsection{IBM Blue Gene/P}
	
	\begin{table}[!h]
		\begin{minipage}{1\linewidth}
			\centering
			\scalebox{0.65}{
				\begin{tabular}{|p{2.1cm}|p{2.3cm}|p{2cm}|p{2.3cm}|p{2.8cm}|p{2.9cm}|p{2.8cm}|p{2.8cm}|}
					\hline
					Число процессоров & Число точек сетки & Время (mpi) & Время (mpi+openMP) & Ускорение (mpi) & Ускорение (mpi+openMP) &  Ускорение (mpi+openMP относительно mpi) & Невязка  \\
					\hline
					128 & $512^3$ &  69.535 & 30.1781  & --- & --- & 2.30 & 0.00000917 \\
					256 & $512^3$ &  34.7832 & 15.107 & 2.0 & 2.0 & 2.30 & 0.00000917 \\
					512 & $512^3$ &  17.5317 & 7.73838 & 2.0 & 1.9 & 2.26 & 0.00000917 \\
					\hline
					128 & $1024^3$ &  553.116 & 234.273 & --- & --- & 2.36 & 0.00000870 \\
					256 & $1024^3$ &  279.237 & 117.414 & 2.0 & 2.0 & 2.37 & 0.00000870 \\
					512 & $1024^3$ &  139.2 & 58.9095 & 2.0 & 2.0 & 2.36 & 0.00000870 \\
					\hline
					128 & $1536^3$ &  838.116 & 356.521 & --- & --- & 2.35 & 0.00000626 \\
					256 & $1536^3$ &  578.233 & 246.805 & 1.5 & 1.4 & 2.35 &  0.00000626 \\
					512 & $1536^3$ &  281.365 & 119.228 & 2.0 & 2.0 & 2.35 &  0.00000626 \\
					\hline
				\end{tabular}
			}
		\end{minipage}%
		\caption{Результаты экспериментов IBM Blue Gene/P}
		\label{table:bluegene}
	\end{table}

	\section{Выводы}
	C увеличением числа процессоров в два раза время выполнения на обоих машинх уменьшается в 2 раза. Это свидетельствует о наличии линейной зависимости между этими величинами. Использование OpenMP позволяет ускорить выполнение расчетов как правило в 2.3 раза. Значение невязки уменьшается с увеличением размера сетки и очевидно никак не зависит от числа процессоров.   
	

\end{document}
